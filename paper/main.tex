\documentclass[12pt, letterpaper]{article}
\usepackage[utf8]{inputenc}
 
\title{Synthetic Accessibility Evolutionary Algorithm Paper}
\author{Steven Bennett, Kim E. Jelfs}
\date{Paper Draft November 2019}

\begin{document}

\maketitle

\section{Abstract}
Using a database filtered for synthetically viable reactions, porous organic cages were
systematically constructed using stk, and its associated evolutionary algorithm, the
chemical space of cages were explored. Using training data obtained from an expert, synthetic
accessibility was defined as a classification problem, and trained on the fingerprints of greater
than 10,000 molecules. Feature selection was used to identify the greatest contributing factors
to synthetic accessibility. Finally, three several synthetic accessibility models were used as part
of the fitness function in the evolutionary algorithm to filter unsynthesisable, ensuring the proposed
porous organic cages are synthetically accessible.

\section{Introduction}
One of the main bottlenecks in the high-throughput materials screening workflow is predicting new materials
that are synthetically viable.

\section{Methods}

\subsection{Classification Model}

\subsection{Evolutionary Algorithm}

\section{Results and Discussion}

\subsection{Shape Persistent Cages}

\subsection{Largest Pore Volume}

\subsection{Identifying Synthesisable Molecules}

\section{Conclusions}
It is possible to envisage a high-throughput experimental synthesis workflow, learning from the results from previous
pathways for synthetically viable compounds.



\end{document}
